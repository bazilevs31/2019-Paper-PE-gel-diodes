%%%%%%%%%%%%%%%%%%%%%%%%%%%%%%%%%%%
%This is the LaTeX ARTICLE template for RSC journals
%Copyright The Royal Society of Chemistry 2016
%%%%%%%%%%%%%%%%%%%%%%%%%%%%%%%%%%%

\documentclass[twoside,twocolumn,9pt]{article}
\usepackage{extsizes}
\usepackage[super,sort&compress,comma]{natbib} 
\usepackage[version=3]{mhchem}
\usepackage[left=1.5cm, right=1.5cm, top=1.785cm, bottom=2.0cm]{geometry}
\usepackage{balance}
\usepackage{times,mathptmx}
\usepackage{sectsty}
\usepackage{graphicx} 
\usepackage{lastpage}
\usepackage[format=plain,justification=justified,singlelinecheck=false,font={stretch=1.125,small,sf},labelfont=bf,labelsep=space]{caption}
\usepackage{float}
\usepackage{fancyhdr}
\usepackage{fnpos}
\usepackage[english]{babel}
\usepackage{array}
\usepackage{droidsans}
\usepackage{charter}
\usepackage[T1]{fontenc}
\usepackage[usenames,dvipsnames]{xcolor}
\usepackage{setspace}
\usepackage[compact]{titlesec}
%%%Please don't disable any packages in the preamble, as this may cause the template to display incorrectly.%%%


\usepackage{epstopdf}%This line makes .eps figures into .pdf - please comment out if not required.

\definecolor{cream}{RGB}{222,217,201}

\begin{document}

\pagestyle{fancy}
\thispagestyle{plain}
\fancypagestyle{plain}{

%%%HEADER%%%
\fancyhead[C]{\includegraphics[width=18.5cm]{head_foot/header_bar}}
\fancyhead[L]{\hspace{0cm}\vspace{1.5cm}\includegraphics[height=30pt]{head_foot/SM}}
\fancyhead[R]{\hspace{0cm}\vspace{1.7cm}\includegraphics[height=55pt]{head_foot/RSC_LOGO_CMYK}}
\renewcommand{\headrulewidth}{0pt}
}
%%%END OF HEADER%%%

%%%PAGE SETUP - Please do not change any commands within this section%%%
\makeFNbottom
\makeatletter
\renewcommand\LARGE{\@setfontsize\LARGE{15pt}{17}}
\renewcommand\Large{\@setfontsize\Large{12pt}{14}}
\renewcommand\large{\@setfontsize\large{10pt}{12}}
\renewcommand\footnotesize{\@setfontsize\footnotesize{7pt}{10}}
\makeatother

\renewcommand{\thefootnote}{\fnsymbol{footnote}}
\renewcommand\footnoterule{\vspace*{1pt}% 
\color{cream}\hrule width 3.5in height 0.4pt \color{black}\vspace*{5pt}} 
\setcounter{secnumdepth}{5}

\makeatletter 
\renewcommand\@biblabel[1]{#1}            
\renewcommand\@makefntext[1]% 
{\noindent\makebox[0pt][r]{\@thefnmark\,}#1}
\makeatother 
\renewcommand{\figurename}{\small{Fig.}~}
\sectionfont{\sffamily\Large}
\subsectionfont{\normalsize}
\subsubsectionfont{\bf}
\setstretch{1.125} %In particular, please do not alter this line.
\setlength{\skip\footins}{0.8cm}
\setlength{\footnotesep}{0.25cm}
\setlength{\jot}{10pt}
\titlespacing*{\section}{0pt}{4pt}{4pt}
\titlespacing*{\subsection}{0pt}{15pt}{1pt}
%%%END OF PAGE SETUP%%%

%%%FOOTER%%%
\fancyfoot{}
\fancyfoot[LO,RE]{\vspace{-7.1pt}\includegraphics[height=9pt]{head_foot/LF}}
\fancyfoot[CO]{\vspace{-7.1pt}\hspace{13.2cm}\includegraphics{head_foot/RF}}
\fancyfoot[CE]{\vspace{-7.2pt}\hspace{-14.2cm}\includegraphics{head_foot/RF}}
\fancyfoot[RO]{\footnotesize{\sffamily{1--\pageref{LastPage} ~\textbar  \hspace{2pt}\thepage}}}
\fancyfoot[LE]{\footnotesize{\sffamily{\thepage~\textbar\hspace{3.45cm} 1--\pageref{LastPage}}}}
\fancyhead{}
\renewcommand{\headrulewidth}{0pt} 
\renewcommand{\footrulewidth}{0pt}
\setlength{\arrayrulewidth}{1pt}
\setlength{\columnsep}{6.5mm}
\setlength\bibsep{1pt}
%%%END OF FOOTER%%%

%%%FIGURE SETUP - please do not change any commands within this section%%%
\makeatletter 
\newlength{\figrulesep} 
\setlength{\figrulesep}{0.5\textfloatsep} 

\newcommand{\topfigrule}{\vspace*{-1pt}% 
\noindent{\color{cream}\rule[-\figrulesep]{\columnwidth}{1.5pt}} }

\newcommand{\botfigrule}{\vspace*{-2pt}% 
\noindent{\color{cream}\rule[\figrulesep]{\columnwidth}{1.5pt}} }

\newcommand{\dblfigrule}{\vspace*{-1pt}% 
\noindent{\color{cream}\rule[-\figrulesep]{\textwidth}{1.5pt}} }

\makeatother
%%%END OF FIGURE SETUP%%%

%%%TITLE, AUTHORS AND ABSTRACT%%%
\twocolumn[
  \begin{@twocolumnfalse}
\vspace{3cm}
\sffamily
\begin{tabular}{m{4.5cm} p{13.5cm} }

\includegraphics{head_foot/DOI} & \noindent\LARGE{\textbf{This is the title$^\dag$}} \\%Article title goes here instead of the text "This is the title"
\vspace{0.3cm} & \vspace{0.3cm} \\

 & \noindent\large{Full Name,$^{\ast}$\textit{$^{a}$} Full Name,\textit{$^{b\ddag}$} and Full Name\textit{$^{a}$}} \\%Author names go here instead of "Full name", etc.

\includegraphics{head_foot/dates} & \noindent\normalsize{The abstract should be a single paragraph which summarises the content of the article. Any references in the abstract should be written out in full \textit{e.g.}\ [Surname \textit{et al., Journal Title}, 2000, \textbf{35}, 3523].} \\%The abstrast goes here instead of the text "The abstract should be..."

\end{tabular}

 \end{@twocolumnfalse} \vspace{0.6cm}

  ]
%%%END OF TITLE, AUTHORS AND ABSTRACT%%%

%%%FONT SETUP - please do not change any commands within this section
\renewcommand*\rmdefault{bch}\normalfont\upshape
\rmfamily
\section*{}
\vspace{-1cm}


%%%FOOTNOTES%%%

\footnotetext{\textit{$^{a}$~Address, Address, Town, Country. Fax: XX XXXX XXXX; Tel: XX XXXX XXXX; E-mail: xxxx@aaa.bbb.ccc}}
\footnotetext{\textit{$^{b}$~Address, Address, Town, Country. }}

%Please use \dag to cite the ESI in the main text of the article.
%If you article does not have ESI please remove the the \dag symbol from the title and the footnotetext below.
\footnotetext{\dag~Electronic Supplementary Information (ESI) available: [details of any supplementary information available should be included here]. See DOI: 10.1039/cXsm00000x/}
%additional addresses can be cited as above using the lower-case letters, c, d, e... If all authors are from the same address, no letter is required

\footnotetext{\ddag~Additional footnotes to the title and authors can be included \textit{e.g.}\ `Present address:' or `These authors contributed equally to this work' as above using the symbols: \ddag, \textsection, and \P. Please place the appropriate symbol next to the author's name and include a \texttt{\textbackslash footnotetext} entry in the the correct place in the list.}


%%%END OF FOOTNOTES%%%

%%%MAIN TEXT%%%%
\section{Introduction}

% Why PE gels are important. What are they? Why do we need to study them more?

With increasing interest in motion capture, soft robotics, and wearable medical technologies, flexible, conductive, and biocompatible electronic devices are required. Unfortunately, most materials currently available are either solid, as in electric wires, or liquid, as in battery electrolytes. Polyelectrolyte gels (PGs) are a promising class of materials that can potentially bridge the gap between current electronic device materials and tomorrow’s soft-material requirements \cite{Huang2018BioinspiredBipolar,Han2009IonicMicrochip,Larson2016HighlySensing}. Polyelectrolyte gels are more flexible than rigid materials (can be stretched up to 3-4x their length) and can be easily contained unlike liquid electrolytes. They can be synthesized to be non toxic, bio friendly and cheap.

% What are gels? How you can make diodes.

Polyelectrolyte gels consist of a backbone like carboxylic acid, water and dissolved salt molecules. In the presence of polar water molecules carboxylic groups dissociate into negatively charged ionized backbone and positive floating counter ions (like protons). When two gels of opposite sign (with positively and negatively charged backbones) are brought in contact they were noticed to act as diodes, effectively  rectifying the current. These materials are referred to as Polyelectrolyte Gel Diodes (PGDs). In the operating regime when the PGD is sandwiched between electrodes water molecules dissociate into protons and hydroxilic groups due to the electrolysis at the electrodes.

% What is the problem here? Why do we need to study it more?

In regular solid state p-n diodes electric current in the entire system is carried by electrons and holes. For the cases of PGDs, however, the electric currents are split into two groups: one carried by the electrons (in  the external circuits and electrodes) and one carried by the ions inside of the gel. 

Regular diode systems have been extensively studied and very well understood. Because of that they have reached high rectifying ratios and can operate in high voltages and currents. PGDs on the other hand, are still not able to operate in the regimes demonstrated by the incumbent methods, despite having superior mechanical properties compared to the incumbents. Largely, this is due to their novelty, lack of understanding and complexity of the phenomena happening inside the material, at the electrodes and external circuits. Hence to improve the materials one needs to carefully study the PGDs. 

% Open the room for studies. Give an overview of types of studies that have been carried out

Up to date most of the studies analyzed the performance of the PGDs by analyzing the behaviour of its constituent ions. These studies could be divided into three main categories: experimental (mainly using I-V current-voltage curve measurements), theoretical (coupled with continuum based computer simulations, employing Poisson-Boltzmann equations) and molecular simulations (studying the behaviour of individual atoms on a molecular level).

% Experimental studies

%Situation:  Experiments. They are good. They can uncover this. Here is a list of experiments that have been carried out. Author 1 did this … -> Complication: Experiments are not enough. More deeper understanding is required. -> Question: What can provide a deeper understanding? —> Answer: Theoretical model provides a more profound description

The authors in several works \cite{So2012IonicElectrodes,Vlassiouk2008NanofluidicSolutions,Zhang2012FlexibleCellulose,Han2009IonicMicrochip} have studied PGDs with variable salt concentration. They found that diodes rectify currents better when there is less salt impurity present. Hence, to maximize the rectification of the ion currents salt impurity must be eliminated. This implies that electrostatic potentials, which are generated by the electric charges of polyelectrolyte gels and their counterions (not the dissociated water at the electrodes), play an important role in the operation of PGDs. Therefore, authors concluded the behaviour of the PGDs  is closely connected to its structure and ion distribution. Although these experimental measurements gave a profound qualitative insight into the gel behaviour, the studies didn't provide quantitative predictions like a theoretical model would do.

% Theoretical studies informed via PB

%Situation: —> Theory. They are good. They can uncover this. Here is a list of theoretical models that have been carried out. Author 1 did this … — > Complication: theoretical models are not enough. Where do they fail? More deeper understanding is required —> Question: What can provide a deeper understanding? —> Answer: molecular model provides a more profound description

 Despite the growing number of experimental observations, there has been limited attempt to theoretically describe the polyelectrolyte gels\cite{Yamamoto2014ElectrochemicalDiodes}. Yamamoto et al.  \cite{Yamamoto2014ElectrochemicalDiodes}  analyzed electrostatic potentials in PGDs to predict the physical mechanisms involved in the ion current rectification of PGDs. To do that, Yamamoto et al. used a simple electrochemical model based on the Poisson-Boltzmann equation.  The authors found that the electrochemical reactions at the gel-electrode interface and applied voltage to main factors that govern the behaviour of the PGDs. Furthermore, authors observed applied voltage to create potential drops, that drive the electrochemical reactions at the gel-electrode interface. The theoretical model employed by the authors managed to quantitatively predict the potential drops, and thus model the rectifying behaviour in the PGD. Although Yamomoto model demonstrates the neccessary physics underlining the PGDs, experimentally measuring the electrostatic potentials drops in PGDs would advance their understanding of the operational mechanisms of PGDs even further. However, experimental measurements of the electrostatic potential drops are cumbersome, hence molecular level computational techniques could be employed to further stress test the Yamamoto model. Additionally, Moy et al.\cite{Moy2000TestsDynamics} has shown that Poisson Boltzmann (PB) used by  Yamamoto et al.\cite{Yamamoto2014ElectrochemicalDiodes} may not be valid throughout the range of parameters that the authors used hence a more detailed analysis like molecular simulations would be needed.



There have been limited attempts to tackle the PGDs through molecular level simulations. Some studies focused on tangential topics like polyelectrolyte sensors \cite{Triandafilidi2018MolecularEffect}, or ionic conductivity of gels \cite{Li2016}. These studies stress-tested the validity of the PB model in different scenarios. Triandafilidi et al. \cite{Triandafilidi2018MolecularEffect} found that the classical Nernst-Donnan model is valid only within certain range of strength of the electrostatic forces (depending on the Bjerrum length $l_B = e^2/(4 \pi \epsilon_0\varepsilon_{diel}k_B T)$) due to the counter-ion condensation. Additionally, Li et al. \cite{Li2016} found that counter-ion condensation may impede the ion transport in the presence of the electrostatic field, hence implying that the classical PB models (like the one used in \cite{Yamamoto2014ElectrochemicalDiodes}) will no longer be valid. One work \cite{Lee2012Grand-canonicalDiode} has investigated the PGDs using Monte-Carlo computer simulations in the Grand Canonical Ensemble. The authors observed that as positive voltage is applied (forward state), small ions start moving across the gel to induce electric current flow. Additionally, with negative voltage (reverse state) applied, current flow became significantly weaker. Despite being in a good agreement with experimental observations, the methodology employed by the authors raises questions about its validity. For example authors employed time dependent variables  for calculating electric current, which in a context of a Monte-Carlo simulation has no physical meaning. This leaves a door open for a deeper more profound understanding of PGDs on a molecular level.


Here we expand on Molecular Dynamics (MD) computer simulation techniques used by \cite{Li2016,Triandafilidi2018MolecularEffect} to understand the behaviour of PGDs on nanoscale.  The present work combines the continuum studies using PB equations with the large-scale Molecular Dynamics(MD) simulations for a "side-by-side" investigation of the PGDs. By matching the PB results with the corresponding MD curves the present work stress-tests the validity of the PB equations used by Yamamoto et al. Furthermore, this work enhances the PB model using the information obtained via MD simulations. The PB model, theoretically informed through the MD simulations could better describe the experimental results and act as a compass for formulating the material to make better electronic device
\section{Theoretical concepts for polyelectrolyte gel diodes}
\label{theor_background}

\subsection{Polyelectrolyte gels}

Many gels are well soluble in water due to an abundance of polar groups along their backbones. In the simplest representation, they consist of an elastic network of crosslinked polyelectrolytes. The elastic network comprises of both charged macroions and neutral monomers. The quantity of charged macroions depends on the degree of ionization $f$. Every charged macroion is compensated by an itinerant counterion to preserve electroneutrality. It seems like counterions would favor to leave the hydrogel and escape into the external solvent as this would increase their translational entropy. However, this would break local electroneutrality. Thus, in case of an immovable gel network the counterions remain within the cell but exert an osmotic pressure onto the surroundings. Besides the mechanical work, entropic forces could potentially break the electroneutrality of the system but only on small length scales $l < R_{ee}$, where $R
_{ee}$ is the average distance between any two crosslinking nodes. The breaking of electroneutrality produces an electrical potential $\Delta \varphi$, also called as Nernst-Donnan potential, which follows readily from equating the chemical potentials of the counterions of charge $q$ inside and outside of the gel:

\begin{equation}\label{eq:donnanpotential}
k_B T \ln (c_o/c_i)= q \Delta \varphi + \Delta \mu_{excess} ,
\end{equation}

where $c_i$ and $c_o$ are the average counterion concentrations inside and outside of the gel and $\mu_{excess}$ is the excess chemical potential due to excluded volume interactions present in the system.



\subsection{Counterion condensation}

At low temperatures, or low dielectric constants electrostatic interactions dominate over thermal energies and classic picture of ion behaviour inside of the gel is no longer valid \cite{Osada2002}.The strength of the electrostatic interaction can be quantified via the Bjerrum length $l_B = k_{Coul} q^2/\varepsilon_{diel} k_B T$, where $\varepsilon_{diel}$ is the dielectric permittivity of the medium, and $k_{Coul}$ is the Coulombic constant. The Bjerrum length corresponds to the distance at which the Coulombic energy equals the thermal energy $k_BT$. According to Manning \cite{Manning_1977}, when $l_B$ becomes larger than the distance $a$ between two consecutive charged monomers, all counterions become bound to their corresponding macroion in a phenomenon called counterion condensation as illustrated in Fig.~\ref{fig:osmpress}B \cite{Mann2005}.

Additionally Manning's theory provides a simple expression for the effective ionization of the gel depending on the Bjerrum length and, hence, the electrostatic strength \cite{Mann2005} becomes 

\begin{equation}\label{eq:mann_explain}
  f_{eff} =
  \begin{cases}
    f & \text{, for $l_b<a$} \\
    f/(l_b/(b f^{-\nu})) & \text{, for $l_b \geq a$}, \\
  \end{cases}
\end{equation}
where $\nu$ is the Flory parameter and is assumed to be $3/5$. In this work, we consider two regimes.  In the first regime at low Bjerrum lengths, the strength of the electrostatic field is lower than the entropic forces. This regime, following \citep{Erbas2015} will be reffered to as weak coupling. Analogously, at high Bjerrum lengths, when the strength of the electrostatic field becomes larger than entropic forces this regime will be referred to as strong coupling. It is worth noting that the gels in external electric fields, may undergo a field-induced de-condensation \citep{Erbas2015}. Hence a careful analysis  has to be done comparing the Manning predictions with the MD simulations.


\subsection{Polyelectrolyte Gel Diodes (PGDs)}

Depending on the sign of the charge of the backbone polyelectrolyte gels could be either positive (comprising of positively charged basic monomer units and negatively charged counter ions) or negative (comprising of negatively charged acidic monomer unit and positive counter ions). When gels of opposite charge brought into contact these gels were shown to rectify the electric current due to the potential drops that are created at the interfaces.

As was shown above due to the break of electroneutrality potential drops appear at the interfaces of the gel diode. When the gel is sandwiched between electrodes and voltage is applied, water present in the gel, undergoes an electrolysis process generating H+ and OH- ions on the cathode and anode respectively. The assymetry of the potential drops give birth to the rectifying ability of the gel. Depending on sign of the voltage applied, the diode will either block the electric current or rectify it. To illustrate that, without loss of generality, lets assume a PGD with posititve gel on the left side and negative gel on the right side. In one case (A), when positive voltage is applied an electrode on the left hand side becomes negatively charged, and the electrode on the right hand side becomes positively charged. This creates an electric field that goes from right to left. This electric field forces positive floating ions on the left side to float to the electrode to minimize the electric field inside of the gel. The abundance of the positive ions blocks the OH- groups from penetrating the diode and recombining with water inside of the gel. Analogues situation happens on the anode. In the case B, when negative voltage is applied (Case B). In this case, iterant ions inside of the gel flock from the electrodes  to the interface between the gel, creating a large potential drop there. This allows and forces the OH- and H+ groups to flock to the interface and recombine.

The modelling of the PGD could be divided into two steps. On the first, the system is modelled with no ions created at the interface due to electrolysis. At this step, only static picture of ion distribution is considered. On the second step, electrolysis is added into consideration. Here the ion creation and reconfiguration is modelled dynamically and currents are analyzed as a function of applied voltage. In the present work, we focus on the first step and carefully investigate the ion distrubution due to electric fields, using Poisson Boltzmann (PB) and Molecular Dynamics (MD) models. 
\subsubsection{Naive Poisson Boltzmann model}

To model the Polyelectrolyte diode consisting of a positive and negative gel sandwiched between the electrodes, we consider only helf of the system due to the characteristic assymetry of the system. In this model, on the right side of the system one has positively charged backbone with concentration of the fixed charges $c_f  = f \frac{3 N_m}{L^3}$ ($f$ is the ionization of the gel) and floating positive and negative counterions with concentration of $c_0$. Assuming the Boltzmann distribution of the iterant ions and constant distribution of the fixed ions, the charge density could be described as:  

$$\frac{d^2 \varphi(z)}{dz^2} = \rho_{charges} = 2  (4 \pi l_b c_0) exp(-\varphi) + 2 (-1) (4 \pi l_b c_0) exp(\varphi) - 4 \pi l_b c_f = 8  \pi l_b c_0 sinh(\varphi(z)) -  4 \pi l_b c_f $$

The parameter $c_0$ is defined, so that $\int_{z=a}^{z=b} c_0 exp(\varphi (z)) = c_f L/2$, where $L$ is the length of the simulation domain. By introducing the inverse Goy-Chapman length $ l_{GC}^{-2} =  k_0^2 = 8 \pi l_b c_0$, as well as $k_f^2 = 4 \pi l_b c_f$, where  $l_b$ is the Bjerrum length.  The PB equation could be rewritten as:

$$\frac{d^2 \varphi(z)}{dz^2} = k_o^2 sinh(\varphi(z)) - k_f^2$$
Note that parameter $c_0$ is not known \textit{a priori} hence  to solve the equation above an iterative approach must be employed. This equation  We will refer to this equation as a Naive PB equation. 


\subsubsection{Modified Poisson Boltzmann model}

In the naive PB model effective ionization does not depend on the Bjerrum length. However as was demonstrated by Triandafilidi et al. citep{Triandafilidi2018} in gels upon certain Bjerrum length countrions undergo counterion condensation and effective ioniztion is decreased. Hence in Modified PB model instead of $k_f$, $c_f$, $f$ only effective parameters need to be used:   $k_{f_{eff}}$, $c_{f_{eff}}$, $f_{eff}$ obtained via substritutng $f$ with $f_{eff}$ .

\section{Methods}
\label{methods}

\subsection{Poisson-Boltzmann}

To solve PB model a finite-difference method is used. The equation is discretizied into $N=1000$ points on the half-domain. Mixed boundary conditions are employed. On the left side of the right half-domain ($z=0$) Dirichlet boundary conditions are employed so the electrical potential is  set to zero $\varphi(z=0) = 0$ due to assymetry of the problem. On the right side of the half-domain ($z=L/2$) electrical field is set to specified value $E = E_{set}$. Throughout the present work an open source distribution of the PB solver is used which is adapted from https://github.com/pv/scikits.bvp1lg. The simulation results were checked by comparing the results with an analytical solution for the case of an electrolyte between the electrodes as well as comparison to the Comsol model.


\subsection{Molecular Dynamics}
Our simulation setup comprises of a box of 20x3x3 unit cells forming a cubic lattice of crosslinking nodes. The unit cells have lattice constant $l = 50\sigma$ and are connected by polyelectrolyte chains with $N_m = 100$ monomers. Due to the translational symmetry of the system there are three polymer chains per each cell with monomer number density $c = 3 \cdot 100 / 50^3 = 2.4 \cdot 10^{-3}\sigma^{-3}$. A fraction $f$ of the monomers in each chain are ionized. The ionized monomers on the left side of the box bear a negative charge $q= -1$, whereas the chains on the right side of the box bear a positive charge $q= +1$.  After the network is created, $N_{ci}$ counterions bearing positive and negative charges are inserted into the left and right sides of the box respectively. Monomers in the network are tethered to static nodes at the left and right boundaries of the simulation box. Periodic boundary conditions (PBC) are employed in lateral directions only. Reflective walls are put at $z=0$ and $z=L_z$ to prevent atoms from escaping the box. Several snapshots illustrating the system are shown in Fig.~\ref{fig:snapshots_all}.

The polyelectrolyte chains are modeled using the coarse-grained Kremer-Grest (KG) bead-spring model \cite{Kremer_1990}. Excluded volume interactions are modeled via a purely repulsive Lennard-Jones (LJ) potential, 

\begin{equation}\label{eq:lj_explain}
  U^{LJ}(r_{ij}) =
  \begin{cases}
    4 \varepsilon\left[\left( \dfrac{\sigma}{r_{ij}}\right)^{12} - \left(\dfrac{\sigma}{r_{ij}}\right)^{6}- U(r_c) \right] & \text{,for $r<r_c$} \\
    \ 0 & \text{,for $r \geq r_c$} \\
  \end{cases}
\end{equation}
where $r_c = 2^{1/6}\sigma$. The Lennard-Jones parameters $\varepsilon$ and $\sigma$ correspond to the depth of the potential well and the distance at which the unshifted potential is zero, respectively. Both parameters are identical for all species present in the system (backbone monomers and floating counterions) and are used as units of energy and distance. Bonded interactions are modeled with a nonlinear FENE potential,

\begin{equation}\label{eq:fene_explain}
  U^{FENE}(r) =
  \begin{cases}
    - \dfrac{1}{2}\ k_f r_f^2 \  \ln \left( 1 -  \left(\dfrac{r}{r_f}\right)^2\right) & \text{, for $r<r_f$} \\
    \ \infty & \text{, for $r \geq r_f$} \\
  \end{cases}
\end{equation}
with $k_f =  30 \epsilon/\sigma^2$, $r_f = 1.5 \sigma $ resulting in a mean bond length of $b \sim \sigma$. Ions interact additionally via the Coulomb potential
\begin{equation}\label{eq:coul_explain}
    U^{Coul}(r_{ij}) = k_{Coul} \dfrac{-q^2}{\varepsilon_{diel} r_{ij}} = -k_BT l_B/r_{ij},\text{ for $r<r^{el}_c$},
\end{equation}
To reduce the computational cost, solvent molecules are treated implicitly through the dielectric constant $\varepsilon_{diel}$.  
To further optimize the computational performance, the electrostatic cutoff distance was set to $r^{el}_c = 13 \sigma$, above which longer-range electrostatic interactions are calculated via a Particle-Particle-Particle Mesh (PPPM) Ewald solver for slab systems \cite{Yeh_1999, Hockney_1988}.  All simulations were performed in the $NVT$ ensemble, with a timestep of integration $dt = 0.005$ and a PPPM accuracy of $10^{-3}$. Molecular dynamics runs were performed using the LAMMPS code \cite{Plimpton_1995}. The MD trajectories were visualized using the VMD code \cite{Humphrey_1996} and analyzed using the MDAnalysis package 
\cite{Khoshlessan_2017,Michaud_Agrawal_2011}. The code base employed in the present work is summarized in the PyGels library \cite{Vasilii2018}. All results are reported in reduced simulation units.

To calculate the pressure and energy profiles, the simulation box was divided into $N_{bins}$ parts along the non-periodic z-direction. For every particle in a bin, an individual per-atom virial tensor $\sigma_{ab}^i$ was calculated and averaged across each bin, where $a, b$ could be any of $x, y, z$ and $i$ is the index of the particle.  The per-atom virial tensor in the case of linear polymers with long-range Coulombic interactions is defined as:

\begin{align}
\begin{split}\label{eq:stress_atom}
    \sigma_{ab} ={}& - [ m v_a v_b +
          \frac{1}{2} \sum_{n = 1}^{N_p} (r_{1_a} F_{1_b} + r_{2_a} F_{2_b}) + \\
         & + \frac{1}{2} \sum_{n = 1}^{N_b} (r_{1_a} F_{1_b} + r_{2_a} F_{2_b}) + 
          {\rm Kspace}(r_{i_a},F_{i_b}) ]
\end{split}
\end{align}
The first term is a kinetic energy contribution for the given atom. The second term is a pairwise energy contribution, ${\bf r}_1$ and $\bf{r}_2$ are the positions of the two atoms in the pairwise interaction where $N_p$ is the total number of neighbors, and ${\bf F}_1$ and ${\bf F}_2$ are the forces on the two atoms resulting from the pairwise interaction. The third term is a contribution from the covalent bonds, and the last term accounts for long-range Coulombic interactions. This tensor has units of energy and described in detail by Thomson et al. \citep{thompson2009general}. Two sets of pressure profiles were calculated during the simulation run. A lateral pressure profile via $P^{lat}(z)=-( \sigma_{xx}^i + \sigma_{yy}^i)/ 2V^{z_j}_{bin}$ and the pressure component along the z axis $P^{zz}(z)=-(\sigma_{zz}^i)/ V^{z_j}_{bin}$. During the production run, the pressure profile was recorded every $500$ steps and averaged in post-processing. The energy and concentration profiles were calculated in a similar fashion.

Analogosly, to calculate the electrostatic field profiles the simulation box was divided into $N_{bins}$ parts along the non-periodic z-direction. For every particle in a bin, an individual per-atom Coulombic force $F^i_{coul}$ was calculated and averaged across each bin, where $i$ is the index of the particle. To calculate only the Coulombic component of the force, first a production run has been performed and posititions of every particle was saved every 100 timesteps ($0.5\tau$). Next, a post-processing simulation rerun has been employed where forces and energies of the particles have been calculated. Here, the parameters of the LJ, FENE  interactions were put to zero, so only the Coulombic component was considered. 

To perform the counterion condensation analysis we measured the distances between the gel monomers and counterions. If counterions were within $r_{cut} = 2 \sigma$ of the gel backbone then it was considered to be condensed. The value of $r_{cut}$ was chosen to include the first two peaks of the radial distribution function of the condensed ions. Counterions that were within $r_{cut}$ of more than one backbone monomer were counted only once. The results for the fractions of the condensed atoms were time averaged across the simulation.




\section{Graphics and tables}
\subsection{Graphics}
Graphics should be inserted on the page where they are first mentioned (unless they are equations, which appear in the flow of the text).\cite{Cotton1999}

\begin{figure}[h]
\centering
  \includegraphics[height=3cm]{example1}
  \caption{An example figure caption \textendash\ the image is from the \textit{Soft Matter} cover gallery.}
  \label{fgr:example}
\end{figure}

\begin{figure*}
 \centering
 \includegraphics[height=3cm]{example2}
 \caption{An image from the \textit{Soft Matter} cover gallery, set as a two-column figure.}
 \label{fgr:example2col}
\end{figure*}

\subsection{Tables}
Tables typeset in RSC house style do not include vertical lines. Table footnote symbols are lower-case italic letters and are typeset at the bottom of the table. Table captions do not end in a full point.\cite{Arduengo1992,Eisenstein2005}


\begin{table}[h]
\small
  \caption{\ An example of a caption to accompany a table}
  \label{tbl:example}
  \begin{tabular*}{0.48\textwidth}{@{\extracolsep{\fill}}lll}
    \hline
    Header one (units) & Header two & Header three \\
    \hline
    1 & 2 & 3 \\
    4 & 5 & 6 \\
    7 & 8 & 9 \\
    10 & 11 & 12 \\
    \hline
  \end{tabular*}
\end{table}

Adding notes to tables can be complicated.  Perhaps the easiest method is to generate these manually.\footnote[4]{Footnotes should appear here. These might include comments relevant to but not central to the matter under discussion, limited experimental and spectral data, and crystallographic data.}

\begin{table*}
\small
  \caption{\ An example of a caption to accompany a table \textendash\ table captions do not end in a full point}
  \label{tbl:example2}
  \begin{tabular*}{\textwidth}{@{\extracolsep{\fill}}lllllll}
    \hline
    Header one & Header two & Header three & Header four & Header five & Header six  & Header seven\\
    \hline
    1 & 2 & 3 & 4 & 5 & 6  & 7\\
    8 & 9 & 10 & 11 & 12 & 13 & 14 \\
    15 & 16 & 17 & 18 & 19 & 20 & 21\\
    \hline
  \end{tabular*}
\end{table*}

\section{Equations}

Equations can be typeset inline \textit{e.g.}\ $ y = mx + c$ or displayed with and without numbers:

 \[ A = \pi r^2 \]

\begin{equation}
  \frac{\gamma}{\epsilon x} r^2 = 2r
\end{equation}

You can also put lists into the text. You can have bulleted or numbered lists of almost any kind. 
The \texttt{mhchem} package can also be used so that formulae are easy to input: \texttt{\textbackslash ce\{H2SO4\}} gives \ce{H2SO4}. 

For footnotes in the main text of the article please number the footnotes to avoid duplicate symbols. \textit{e.g.}\ \texttt{\textbackslash footnote[num]\{your text\}}. The corresponding author $\ast$ counts as footnote 1, ESI as footnote 2, \textit{e.g.}\ if there is no ESI, please start at [num]=[2], if ESI is cited in the title please start at [num]=[3] \textit{etc.} Please also cite the ESI within the main body of the text using \dag.

\section{Conclusions}
The conclusions section should come at the end of article. For the reference section, the style file \texttt{rsc.bst} can be used to generate the correct reference style.
%%%END OF MAIN TEXT%%%

%The \balance command can be used to balance the columns on the final page if desired. It should be placed anywhere within the first column of the last page.

\balance

%If notes are included in your references you can change the title from 'References' to 'Notes and references' using the following command:
%\renewcommand\refname{Notes and references}

%%%REFERENCES%%%
\bibliography{rsc2} %You need to replace "rsc" on this line with the name of your .bib file
\bibliographystyle{rsc} %the RSC's .bst file

\end{document}
