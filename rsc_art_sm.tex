%%%%%%%%%%%%%%%%%%%%%%%%%%%%%%%%%%%
%This is the LaTeX ARTICLE template for RSC journals
%Copyright The Royal Society of Chemistry 2016
%%%%%%%%%%%%%%%%%%%%%%%%%%%%%%%%%%%

\documentclass[twoside,twocolumn,9pt]{article}
\usepackage{extsizes}
\usepackage[super,sort&compress,comma]{natbib} 
\usepackage[version=3]{mhchem}
\usepackage[left=1.5cm, right=1.5cm, top=1.785cm, bottom=2.0cm]{geometry}
\usepackage{balance}
\usepackage{times,mathptmx}
\usepackage{sectsty}
\usepackage{graphicx} 
\usepackage{lastpage}
\usepackage[format=plain,justification=justified,singlelinecheck=false,font={stretch=1.125,small,sf},labelfont=bf,labelsep=space]{caption}
\usepackage{float}
\usepackage{fancyhdr}
\usepackage{fnpos}
\usepackage[english]{babel}
\usepackage{array}
\usepackage{droidsans}
\usepackage{charter}
\usepackage[T1]{fontenc}
\usepackage[usenames,dvipsnames]{xcolor}
\usepackage{setspace}
\usepackage[compact]{titlesec}
%%%Please don't disable any packages in the preamble, as this may cause the template to display incorrectly.%%%


\usepackage{epstopdf}%This line makes .eps figures into .pdf - please comment out if not required.

\definecolor{cream}{RGB}{222,217,201}

\begin{document}

\pagestyle{fancy}
\thispagestyle{plain}
\fancypagestyle{plain}{

%%%HEADER%%%
\fancyhead[C]{\includegraphics[width=18.5cm]{head_foot/header_bar}}
\fancyhead[L]{\hspace{0cm}\vspace{1.5cm}\includegraphics[height=30pt]{head_foot/SM}}
\fancyhead[R]{\hspace{0cm}\vspace{1.7cm}\includegraphics[height=55pt]{head_foot/RSC_LOGO_CMYK}}
\renewcommand{\headrulewidth}{0pt}
}
%%%END OF HEADER%%%

%%%PAGE SETUP - Please do not change any commands within this section%%%
\makeFNbottom
\makeatletter
\renewcommand\LARGE{\@setfontsize\LARGE{15pt}{17}}
\renewcommand\Large{\@setfontsize\Large{12pt}{14}}
\renewcommand\large{\@setfontsize\large{10pt}{12}}
\renewcommand\footnotesize{\@setfontsize\footnotesize{7pt}{10}}
\makeatother

\renewcommand{\thefootnote}{\fnsymbol{footnote}}
\renewcommand\footnoterule{\vspace*{1pt}% 
\color{cream}\hrule width 3.5in height 0.4pt \color{black}\vspace*{5pt}} 
\setcounter{secnumdepth}{5}

\makeatletter 
\renewcommand\@biblabel[1]{#1}            
\renewcommand\@makefntext[1]% 
{\noindent\makebox[0pt][r]{\@thefnmark\,}#1}
\makeatother 
\renewcommand{\figurename}{\small{Fig.}~}
\sectionfont{\sffamily\Large}
\subsectionfont{\normalsize}
\subsubsectionfont{\bf}
\setstretch{1.125} %In particular, please do not alter this line.
\setlength{\skip\footins}{0.8cm}
\setlength{\footnotesep}{0.25cm}
\setlength{\jot}{10pt}
\titlespacing*{\section}{0pt}{4pt}{4pt}
\titlespacing*{\subsection}{0pt}{15pt}{1pt}
%%%END OF PAGE SETUP%%%

%%%FOOTER%%%
\fancyfoot{}
\fancyfoot[LO,RE]{\vspace{-7.1pt}\includegraphics[height=9pt]{head_foot/LF}}
\fancyfoot[CO]{\vspace{-7.1pt}\hspace{13.2cm}\includegraphics{head_foot/RF}}
\fancyfoot[CE]{\vspace{-7.2pt}\hspace{-14.2cm}\includegraphics{head_foot/RF}}
\fancyfoot[RO]{\footnotesize{\sffamily{1--\pageref{LastPage} ~\textbar  \hspace{2pt}\thepage}}}
\fancyfoot[LE]{\footnotesize{\sffamily{\thepage~\textbar\hspace{3.45cm} 1--\pageref{LastPage}}}}
\fancyhead{}
\renewcommand{\headrulewidth}{0pt} 
\renewcommand{\footrulewidth}{0pt}
\setlength{\arrayrulewidth}{1pt}
\setlength{\columnsep}{6.5mm}
\setlength\bibsep{1pt}
%%%END OF FOOTER%%%

%%%FIGURE SETUP - please do not change any commands within this section%%%
\makeatletter 
\newlength{\figrulesep} 
\setlength{\figrulesep}{0.5\textfloatsep} 

\newcommand{\topfigrule}{\vspace*{-1pt}% 
\noindent{\color{cream}\rule[-\figrulesep]{\columnwidth}{1.5pt}} }

\newcommand{\botfigrule}{\vspace*{-2pt}% 
\noindent{\color{cream}\rule[\figrulesep]{\columnwidth}{1.5pt}} }

\newcommand{\dblfigrule}{\vspace*{-1pt}% 
\noindent{\color{cream}\rule[-\figrulesep]{\textwidth}{1.5pt}} }

\makeatother
%%%END OF FIGURE SETUP%%%

%%%TITLE, AUTHORS AND ABSTRACT%%%
\twocolumn[
  \begin{@twocolumnfalse}
\vspace{3cm}
\sffamily
\begin{tabular}{m{4.5cm} p{13.5cm} }

\includegraphics{head_foot/DOI} & \noindent\LARGE{\textbf{This is the title$^\dag$}} \\%Article title goes here instead of the text "This is the title"
\vspace{0.3cm} & \vspace{0.3cm} \\

 & \noindent\large{Full Name,$^{\ast}$\textit{$^{a}$} Full Name,\textit{$^{b\ddag}$} and Full Name\textit{$^{a}$}} \\%Author names go here instead of "Full name", etc.

\includegraphics{head_foot/dates} & \noindent\normalsize{The abstract should be a single paragraph which summarises the content of the article. Any references in the abstract should be written out in full \textit{e.g.}\ [Surname \textit{et al., Journal Title}, 2000, \textbf{35}, 3523].} \\%The abstrast goes here instead of the text "The abstract should be..."

\end{tabular}

 \end{@twocolumnfalse} \vspace{0.6cm}

  ]
%%%END OF TITLE, AUTHORS AND ABSTRACT%%%

%%%FONT SETUP - please do not change any commands within this section
\renewcommand*\rmdefault{bch}\normalfont\upshape
\rmfamily
\section*{}
\vspace{-1cm}


%%%FOOTNOTES%%%

\footnotetext{\textit{$^{a}$~Address, Address, Town, Country. Fax: XX XXXX XXXX; Tel: XX XXXX XXXX; E-mail: xxxx@aaa.bbb.ccc}}
\footnotetext{\textit{$^{b}$~Address, Address, Town, Country. }}

%Please use \dag to cite the ESI in the main text of the article.
%If you article does not have ESI please remove the the \dag symbol from the title and the footnotetext below.
\footnotetext{\dag~Electronic Supplementary Information (ESI) available: [details of any supplementary information available should be included here]. See DOI: 10.1039/cXsm00000x/}
%additional addresses can be cited as above using the lower-case letters, c, d, e... If all authors are from the same address, no letter is required

\footnotetext{\ddag~Additional footnotes to the title and authors can be included \textit{e.g.}\ `Present address:' or `These authors contributed equally to this work' as above using the symbols: \ddag, \textsection, and \P. Please place the appropriate symbol next to the author's name and include a \texttt{\textbackslash footnotetext} entry in the the correct place in the list.}


%%%END OF FOOTNOTES%%%

%%%MAIN TEXT%%%%
\section{Introduction}

% Why PE gels are important. What are they? Why do we need to study them more?

With increasing interest in motion capture, soft robotics, and wearable medical technologies, flexible, conductive, and biocompatible electronic devices are required. Unfortunately, most materials currently available are either solid, as in electric wires, or liquid, as in battery electrolytes. Polyelectrolyte gels (PGs) are a promising class of materials that can potentially bridge the gap between current electronic device materials and tomorrow’s soft-material requirements \cite{Huang2018BioinspiredBipolar,Han2009IonicMicrochip,Larson2016HighlySensing}. Polyelectrolyte gels are more flexible than rigid materials (can be stretched up to 3-4x their length) and can be easily contained unlike liquid electrolytes. They can be synthesized to be non toxic, bio friendly and cheap.

% What are gels? How you can make diodes.

Polyelectrolyte gels consist of a backbone like carboxylic acid, water and dissolved salt molecules. In the presence of polar water molecules carboxylic groups dissociate into negatively charged ionized backbone and positive floating counter ions (like protons). When two gels of opposite sign (with positively and negatively charged backbones) are brought in contact they were noticed to act as diodes, effectively  rectifying the current. These materials are referred to as Polyelectrolyte Gel Diodes (PGDs). In the operating regime when the PGD is sandwiched between electrodes water molecules dissociate into protons and hydroxilic groups due to the electrolysis at the electrodes.

% What is the problem here? Why do we need to study it more?

In regular solid state p-n diodes electric current in the entire system is carried by electrons and holes. For the cases of PGDs, however, the electric currents are split into two groups: one carried by the electrons (in  the external circuits and electrodes) and one carried by the ions inside of the gel. 

Regular diode systems have been extensively studied and very well understood. Because of that they have reached high rectifying ratios and can operate in high voltages and currents. PGDs on the other hand, are still not able to operate in the regimes demonstrated by the incumbent methods, despite having superior mechanical properties compared to the incumbents. Largely, this is due to their novelty, lack of understanding and complexity of the phenomena happening inside the material, at the electrodes and external circuits. Hence to improve the materials one needs to carefully study the PGDs. 

% Open the room for studies. Give an overview of types of studies that have been carried out

Up to date most of the studies analyzed the performance of the PGDs by analyzing the behaviour of its constituent ions. These studies could be divided into three main categories: experimental (mainly using I-V current-voltage curve measurements), theoretical (coupled with continuum based computer simulations, employing Poisson-Boltzmann equations) and molecular simulations (studying the behaviour of individual atoms on a molecular level).

% Experimental studies

%Situation:  Experiments. They are good. They can uncover this. Here is a list of experiments that have been carried out. Author 1 did this … -> Complication: Experiments are not enough. More deeper understanding is required. -> Question: What can provide a deeper understanding? —> Answer: Theoretical model provides a more profound description

The authors in several works \cite{So2012IonicElectrodes,Vlassiouk2008NanofluidicSolutions,Zhang2012FlexibleCellulose,Han2009IonicMicrochip} have studied PGDs with variable salt concentration. They found that diodes rectify currents better when there is less salt impurity present. Hence, to maximize the rectification of the ion currents salt impurity must be eliminated. This implies that electrostatic potentials, which are generated by the electric charges of polyelectrolyte gels and their counterions (not the dissociated water at the electrodes), play an important role in the operation of PGDs. Therefore, authors concluded the behaviour of the PGDs  is closely connected to its structure and ion distribution. Although these experimental measurements gave a profound qualitative insight into the gel behaviour, the studies didn't provide quantitative predictions like a theoretical model would do.

% Theoretical studies informed via PB

%Situation: —> Theory. They are good. They can uncover this. Here is a list of theoretical models that have been carried out. Author 1 did this … — > Complication: theoretical models are not enough. Where do they fail? More deeper understanding is required —> Question: What can provide a deeper understanding? —> Answer: molecular model provides a more profound description

 Despite the growing number of experimental observations, there has been limited attempt to theoretically describe the polyelectrolyte gels\cite{Yamamoto2014ElectrochemicalDiodes}. Yamamoto et al.  \cite{Yamamoto2014ElectrochemicalDiodes}  analyzed electrostatic potentials in PGDs to predict the physical mechanisms involved in the ion current rectification of PGDs. To do that, Yamamoto et al. used a simple electrochemical model based on the Poisson-Boltzmann equation.  The authors found that the electrochemical reactions at the gel-electrode interface and applied voltage to main factors that govern the behaviour of the PGDs. Furthermore, authors observed applied voltage to create potential drops, that drive the electrochemical reactions at the gel-electrode interface. The theoretical model employed by the authors managed to quantitatively predict the potential drops, and thus model the rectifying behaviour in the PGD. Although Yamomoto model demonstrates the neccessary physics underlining the PGDs, experimentally measuring the electrostatic potentials drops in PGDs would advance their understanding of the operational mechanisms of PGDs even further. However, experimental measurements of the electrostatic potential drops are cumbersome, hence molecular level computational techniques could be employed to further stress test the Yamamoto model. Additionally, Moy et al.\cite{Moy2000TestsDynamics} has shown that Poisson Boltzmann (PB) used by  Yamamoto et al.\cite{Yamamoto2014ElectrochemicalDiodes} may not be valid throughout the range of parameters that the authors used hence a more detailed analysis like molecular simulations would be needed.



There have been limited attempts to tackle the PGDs through molecular level simulations. Some studies focused on tangential topics like polyelectrolyte sensors \cite{Triandafilidi2018MolecularEffect}, or ionic conductivity of gels \cite{Li2016}. These studies stress-tested the validity of the PB model in different scenarios. Triandafilidi et al. \cite{Triandafilidi2018MolecularEffect} found that the classical Nernst-Donnan model is valid only within certain range of strength of the electrostatic forces (depending on the Bjerrum length $l_B = e^2/(4 \pi \epsilon_0\varepsilon_{diel}k_B T)$) due to the counter-ion condensation. Additionally, Li et al. \cite{Li2016} found that counter-ion condensation may impede the ion transport in the presence of the electrostatic field, hence implying that the classical PB models (like the one used in \cite{Yamamoto2014ElectrochemicalDiodes}) will no longer be valid. One work \cite{Lee2012Grand-canonicalDiode} has investigated the PGDs using Monte-Carlo computer simulations in the Grand Canonical Ensemble. The authors observed that as positive voltage is applied (forward state), small ions start moving across the gel to induce electric current flow. Additionally, with negative voltage (reverse state) applied, current flow became significantly weaker. Despite being in a good agreement with experimental observations, the methodology employed by the authors raises questions about its validity. For example authors employed time dependent variables  for calculating electric current, which in a context of a Monte-Carlo simulation has no physical meaning. This leaves a door open for a deeper more profound understanding of PGDs on a molecular level.


Here we expand on Molecular Dynamics (MD) computer simulation techniques used by \cite{Li2016,Triandafilidi2018MolecularEffect} to understand the behaviour of PGDs on nanoscale.  The present work combines the continuum studies using PB equations with the large-scale Molecular Dynamics(MD) simulations for a "side-by-side" investigation of the PGDs. By matching the PB results with the corresponding MD curves the present work stress-tests the validity of the PB equations used by Yamamoto et al. Furthermore, this work enhances the PB model using the information obtained via MD simulations. The PB model, theoretically informed through the MD simulations could better describe the experimental results and act as a compass for formulating the material to make better electronic devices.


\section{Methods}

\section{Graphics and tables}
\subsection{Graphics}
Graphics should be inserted on the page where they are first mentioned (unless they are equations, which appear in the flow of the text).\cite{Cotton1999}

\begin{figure}[h]
\centering
  \includegraphics[height=3cm]{example1}
  \caption{An example figure caption \textendash\ the image is from the \textit{Soft Matter} cover gallery.}
  \label{fgr:example}
\end{figure}

\begin{figure*}
 \centering
 \includegraphics[height=3cm]{example2}
 \caption{An image from the \textit{Soft Matter} cover gallery, set as a two-column figure.}
 \label{fgr:example2col}
\end{figure*}

\subsection{Tables}
Tables typeset in RSC house style do not include vertical lines. Table footnote symbols are lower-case italic letters and are typeset at the bottom of the table. Table captions do not end in a full point.\cite{Arduengo1992,Eisenstein2005}


\begin{table}[h]
\small
  \caption{\ An example of a caption to accompany a table}
  \label{tbl:example}
  \begin{tabular*}{0.48\textwidth}{@{\extracolsep{\fill}}lll}
    \hline
    Header one (units) & Header two & Header three \\
    \hline
    1 & 2 & 3 \\
    4 & 5 & 6 \\
    7 & 8 & 9 \\
    10 & 11 & 12 \\
    \hline
  \end{tabular*}
\end{table}

Adding notes to tables can be complicated.  Perhaps the easiest method is to generate these manually.\footnote[4]{Footnotes should appear here. These might include comments relevant to but not central to the matter under discussion, limited experimental and spectral data, and crystallographic data.}

\begin{table*}
\small
  \caption{\ An example of a caption to accompany a table \textendash\ table captions do not end in a full point}
  \label{tbl:example2}
  \begin{tabular*}{\textwidth}{@{\extracolsep{\fill}}lllllll}
    \hline
    Header one & Header two & Header three & Header four & Header five & Header six  & Header seven\\
    \hline
    1 & 2 & 3 & 4 & 5 & 6  & 7\\
    8 & 9 & 10 & 11 & 12 & 13 & 14 \\
    15 & 16 & 17 & 18 & 19 & 20 & 21\\
    \hline
  \end{tabular*}
\end{table*}

\section{Equations}

Equations can be typeset inline \textit{e.g.}\ $ y = mx + c$ or displayed with and without numbers:

 \[ A = \pi r^2 \]

\begin{equation}
  \frac{\gamma}{\epsilon x} r^2 = 2r
\end{equation}

You can also put lists into the text. You can have bulleted or numbered lists of almost any kind. 
The \texttt{mhchem} package can also be used so that formulae are easy to input: \texttt{\textbackslash ce\{H2SO4\}} gives \ce{H2SO4}. 

For footnotes in the main text of the article please number the footnotes to avoid duplicate symbols. \textit{e.g.}\ \texttt{\textbackslash footnote[num]\{your text\}}. The corresponding author $\ast$ counts as footnote 1, ESI as footnote 2, \textit{e.g.}\ if there is no ESI, please start at [num]=[2], if ESI is cited in the title please start at [num]=[3] \textit{etc.} Please also cite the ESI within the main body of the text using \dag.

\section{Conclusions}
The conclusions section should come at the end of article. For the reference section, the style file \texttt{rsc.bst} can be used to generate the correct reference style.
%%%END OF MAIN TEXT%%%

%The \balance command can be used to balance the columns on the final page if desired. It should be placed anywhere within the first column of the last page.

\balance

%If notes are included in your references you can change the title from 'References' to 'Notes and references' using the following command:
%\renewcommand\refname{Notes and references}

%%%REFERENCES%%%
\bibliography{rsc2} %You need to replace "rsc" on this line with the name of your .bib file
\bibliographystyle{rsc} %the RSC's .bst file

\end{document}
